\section[Section]{Three Stage Thinking}

\begin{frame}
    \frametitle{Three Stage Thinking}
    \begin{itemize}
        \setlength\itemsep{3em}
        \item Edit
        \item Add
        \item Commit
    \end{itemize}
\end{frame}

\begin{frame}
    \frametitle{Three Stage Thinking}
    \alert{Demo 3}: Three Stage Thinking
\end{frame}
\note[itemize]{
    \item \alert{Demo 3}: Three Stage Thinking
}

\begin{frame}
    \frametitle{Lesson 1}
    \alert{Lesson 1}: Three Stage Thinking
\end{frame}
\note[itemize]{
    \item \alert{Lesson 1}: Three Stage Thinking
}

\begin{frame}
    \frametitle{Commit Messages}
    The Technical Bits
    \begin{itemize}
        \item Short (aim for 50 characters or less) summary
        \item Followed by a blank line
        \item Body wrapped to 72 characters
    \end{itemize}
\end{frame}
\note[itemize]{
    \item \alert{Webpage}: \url{http://tbaggery.com/2008/04/19/a-note-about-git-commit-messages.html}
    \item Summary line must be separated by a blank space or many tools get a
          little confused
    \item Some views truncate the summary line; soft 50; hard 72
    \item Hard wrap body to 72 characters
    \item \texttt{git log}, \texttt{git format-patch}, etc do not wrap message
}

\begin{frame}
    \frametitle{Commit Messages}
    The Conventional Bits
    \begin{itemize}
        \item Make your commits atomic
        \item Justify your changes; write detailed messages
        \item Write is the imperative: "Fix bug" and not "Fixed bug" or
              "Fixes bug"
        \item Present tense for current commit
        \item Past tense for earlier commits
        \item Future tense for later commits
        \item No period on subject line
        \item Meta-data at the bottom
    \end{itemize}
\end{frame}
\note[itemize]{
    \item These are common expectations
    \item Like most social conventions, will be used to judge you more so than
          that they are technically superior
    \item I include them here so that you can understand and fit in
    \item Atomic: words like and, also, consider splitting commit
    \item Justify: open-source mailing list mentality; what, why, how
    \item Imperative style dates back to, at least, GNU changelogs
    \item Meta-data at the bottom: signed-of-by, change-id, issue tracker
    \item Look at a linux kernel log
}

\begin{frame}
    \frametitle{12 Everyday Commands}
    \begin{multicols}{3}
        \begin{itemize}
            \setlength\itemsep{3em}
            \item \alert{add}
            \item branch
            \item checkout
            \item \alert{commit}
            \item \alert{diff}
            \item fetch
            \item \alert{help}
            \item \alert{log}
            \item merge
            \item push
            \item rebase
            \item \alert{status}
        \end{itemize}
    \end{multicols}
\end{frame}
\note[itemize]{
    \item You've already seen these
}
