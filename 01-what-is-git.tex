\section[Section]{What is Git?}

\tikzset{
    box/.style={
        draw=solarized-base01,
        fill=solarized-base3!50,
        %thick,
        drop shadow={
            opacity=0.15,
        },
    },
    box2/.style={
        draw=solarized-base01,
        fill=solarized-base2!50,
        %thick,
        drop shadow={
            opacity=0.15,
        },
    },
    arrow/.style={
        %semithick,
        Latex-,
        draw=gray,
    },
    file/.style={
        shape=document,
        minimum height=4em,
        minimum width=3em,
        draw=solarized-base01,
        fill=solarized-blue!20,
        %thick,
        drop shadow={
            opacity=0.15,
        },
        font=\fontfamily{lmtt}\selectfont\small,
    },
    version/.style={
        shape=rounded rectangle,
        rounded rectangle arc length=90,
        minimum height=1.6em,
        minimum width=2em,
        draw=solarized-base01,
        fill=solarized-green!20,
        %very thick,
        drop shadow={
            opacity=0.15,
        },
        font=\fontfamily{lmtt}\selectfont\small,
    },
}

\newsavebox{\versiondatabase}
\savebox{\versiondatabase}{%
    \begin{tikzpicture}[node distance=1em]
        \pgfdeclarelayer{background}
        \pgfsetlayers{background,main}
        \node (vdb) [] {Version Database};
        \node (v3) [below=of vdb, version] {Version 3};
        \node (v2) [below=of v3, version] {Version 2};
        \node (v1) [below=of v2, version] {Version 1};
        \draw [arrow] (v2) -- (v3);
        \draw [arrow] (v1) -- (v2);
        \begin{pgfonlayer}{background}
            \node (vdbbox) [fit=(vdb)(v3)(v2)(v1),box2] {};
        \end{pgfonlayer}
    \end{tikzpicture}%
}

\newsavebox{\files}
\savebox{\files}{%
    \begin{tikzpicture}[]
        \foreach \i in {1,2,3} {
            \begin{scope}[shift={(.1*\i,.1*\i)}]
                \node () [file] {Files};
            \end{scope}
        }
    \end{tikzpicture}%
}

\begin{frame}
    \frametitle{What is Git?}
    \centering
    \minibox{
        git \textit{noun} \textbackslash'git\textbackslash\\
        \textit{British}\\
        \qquad : a foolish or worthless person
    }
\end{frame}

\begin{frame}
    \frametitle{What is Git?}
    \chapquote{``I'm an egotistical bastard, and I name all my projects after
               myself. First 'Linux', now 'Git'"}{Linux Torvalds}
\end{frame}

\begin{frame}
    \frametitle{What is Git?}
    Random three-letter combination that is pronounceable, and not actually
    used by any common UNIX command. The fact that it is a mispronunciation of
    "get" may or may not be relevant.
\end{frame}

\begin{frame}
    \frametitle{What is Git?}
    \centering
    Git is an open source,\\
    distributed version control system\\
    designed for speed and efficiency
\end{frame}
\note{Official tagline of Git}

\begin{frame}
    \frametitle{What is Git?}
    \centering
    Git is an \alert{open source},\\
    distributed version control system\\
    designed for speed and efficiency
\end{frame}
\note[itemize]{
    \item LGPL-2.1
    \item https://github.com/git/git
    \item First release was 7 April 2005 from Linus Torvalds
    \item Maintained by Junio Hamano since since 26 July 2005
    \item Kernel hacker mentality. It was written to manage the Linux
          kernel. So, it won't stop you from shooting yourself in the foot.
}

\begin{frame}
    \frametitle{What is Git?}
    \centering
    Git is an open source,\\
    \alert{distributed} version control system\\
    designed for speed and efficiency
\end{frame}

\begin{frame}
    \frametitle{Centralized Version Control}
    \centering
    \begin{tikzpicture}[node distance=1em]
        \pgfdeclarelayer{background}
        \pgfsetlayers{background,main}
        \node (server) [] {Server};
        \node (vdbbox) [below=of server,inner sep=0] {\usebox{\versiondatabase}};
        \node (client1) [] at (-4,-2) {Client A};
        \node (files1) [below=of client1,inner sep=0] {\usebox{\files}};
        \node (client2) [] at (4,-2) {Client B};
        \node (files2) [below=of client2,inner sep=0] {\usebox{\files}};
        \begin{pgfonlayer}{background}
            \node [fit=(server)(vdbbox),box] {};
            \node [fit=(client1)(files1),box] {};
            \node [fit=(client2)(files2),box] {};
        \end{pgfonlayer}
        \draw [arrow] (vdbbox) -- (files1);
        \draw [arrow] (files1) -- (vdbbox);
        \draw [arrow] (vdbbox) -- (files2);
        \draw [arrow] (files2) -- (vdbbox);
    \end{tikzpicture}
\end{frame}
\note[itemize]{
    \item Client, server model
    \item Version database on only one server
    \item Download a snapshot
    \item Send incremental changes to the server
    \item Division of responsibility; some things only server can do, some
          things only client can do.
}

\begin{frame}
    \frametitle{Distributed Version Control}
    \begin{adjustbox}{max totalsize={\textwidth}{\textheight},center}
        \begin{tikzpicture}[node distance=1.4em]
            \pgfdeclarelayer{background}
            \pgfsetlayers{background,main}
            \node (server) [] {Server};
            \node (vdbbox) [below=of server,inner sep=0] {\usebox{\versiondatabase}};
            \node (client1) [] at (-4,0) {Client A};
            \node (files1) [below=of client1,inner sep=0] {\usebox{\files}};
            \node (vdbbox1) [below=of files1,inner sep=0] {\usebox{\versiondatabase}};
            \node (client2) [] at (4,0) {Client B};
            \node (files2) [below=of client2,inner sep=0] {\usebox{\files}};
            \node (vdbbox2) [below=of files2,inner sep=0] {\usebox{\versiondatabase}};
            \begin{pgfonlayer}{background}
                \node [fit=(server)(vdbbox),box] {};
                \node [fit=(client1)(files1)(vdbbox1),box] {};
                \node [fit=(client2)(files2)(vdbbox2),box] {};
            \end{pgfonlayer}
            \draw [arrow] (vdbbox) -- (vdbbox1);
            \draw [arrow] (vdbbox1) -- (vdbbox);
            \draw [arrow] (files1) -- (vdbbox1);
            \draw [arrow] (vdbbox1) -- (files1);
            \draw [arrow] (vdbbox) -- (vdbbox2);
            \draw [arrow] (vdbbox2) -- (vdbbox);
            \draw [arrow] (files2) -- (vdbbox2);
            \draw [arrow] (vdbbox2) -- (files2);
            \draw [arrow] ([yshift=-3em] vdbbox1.east) -- ([yshift=-3em] vdbbox2.west);
            \draw [arrow] ([yshift=-3em] vdbbox2.west) -- ([yshift=-3em] vdbbox1.east);
        \end{tikzpicture}
    \end{adjustbox}
\end{frame}
\note[itemize]{
    \item Peer to peer
    \item Version database on every machine
    \item Clients can talk to each other
    \item Download the entire repository
    \item Operate locally, share explicitely
    \item You can have a central server
    \item Servers are only different in that, as an optimization, they
          don't have working copies of the files. The clients are actually
          more featureful than the server.
}

\begin{frame}
    \frametitle{What is Git?}
    \centering
    Git is an open source,\\
    distributed version control system\\
    designed for \alert{speed and efficiency}
\end{frame}

\newcommand{\gitsvnplot}[4]{%
    \begin{tikzpicture}[baseline]
        \ifthenelse{\equal{#1}{true}}{
            \begin{axis}[
                ybar=0,
                bar width=1.2cm,
                x=1.2cm,
                enlarge x limits={abs=0.1cm},
                ymin=0,
                symbolic x coords={#2},
                xtick=data,
                yticklabels={,,},
                tick style={draw=none},
                axis x line*=bottom,
                xticklabel style={
                    font=\small,
                },
                legend style={
                    %at={(1.5,1)},
                    fill=solarized-base3!40,
                    %draw=none,
                    draw=solarized-base01,
                },
                legend columns=-1,
                legend to name=foobar,
                legend entries={Git,Subversion},
                nodes near coords,
                nodes near coords={\pgfmathprintnumber[fixed,precision=2,zerofill=true]{\pgfplotspointmeta}},
                axis line style={solarized-base01},
                axis y line*=left,
                ylabel={Seconds},
                ylabel near ticks,
            ]
        }{
            \begin{axis}[
                ybar=0,
                bar width=1.2cm,
                x=1.2cm,
                enlarge x limits={abs=0.1cm},
                ymin=0,
                symbolic x coords={#2},
                xtick=data,
                yticklabels={,,},
                tick style={draw=none},
                axis x line*=bottom,
                xticklabel style={
                    font=\small,
                },
                legend style={
                    %at={(1.5,1)},
                    fill=solarized-base3!40,
                    %draw=none,
                    draw=solarized-base01,
                },
                legend columns=-1,
                legend to name=foobar,
                legend entries={Git,Subversion},
                nodes near coords,
                nodes near coords={\pgfmathprintnumber[fixed,precision=2,zerofill=true]{\pgfplotspointmeta}},
                scaled y ticks=false,
                axis line style={solarized-base01},
                hide y axis,
            ]
        }
            \addplot[solarized-orange,fill=solarized-orange!20] coordinates {(#2,#3)};
            \addplot[solarized-blue,fill=solarized-blue!20] coordinates {(#2,#4)};
        \end{axis}
    \end{tikzpicture}%
}
\begin{frame}
    \frametitle{Git is Fast}
    \begin{adjustbox}{max totalsize={\textwidth}{.8\textheight},center}
        \begin{tabular}{llllll}
            \gitsvnplot{true}{Commit Files}{0.64}{2.60} &
            \gitsvnplot{false}{Commit Images}{1.53}{24.70} &
            \gitsvnplot{false}{Diff Current}{0.25}{1.09} &
            \gitsvnplot{false}{Diff Recent}{0.25}{3.99} &
            \gitsvnplot{false}{Diff Tags}{1.17}{83.57} &
            \gitsvnplot{false}{Clone}{107.5}{14.0} \\
            \gitsvnplot{true}{Log (50)}{0.01}{0.38} &
            \gitsvnplot{false}{Log (All)}{0.52}{169.20} &
            \gitsvnplot{false}{Log (File)}{0.60}{82.84} &
            \gitsvnplot{false}{Update}{0.90}{2.82} &
            \gitsvnplot{false}{Blame}{1.91}{3.04} &
            \gitsvnplot{false}{Size}{181.0}{132.0} \\
            \multicolumn{6}{r}{\ref{foobar}} \\
        \end{tabular}
    \end{adjustbox}
\end{frame}
\note[itemize]{
    \item Data from Scott Chacon http://git-scm.com/about/small-and-fast
    \item All operations are local except explicit synchronization
    \item No network access needed to:
    \begin{itemize}
        \item Perform a diff
        \item View file history
        \item Commit changes
        \item Merge branches
        \item Switch branches
        \item Checkout another revision
        \item Blame a file
        \item Search for the change that introduced a bug
    \end{itemize}
}

\begin{frame}
    \frametitle{What is Git?}
    \centering
    \alert{Immutable} \\
    (almost) never removes data
\end{frame}
\note[itemize]{
    \item You will hear about rewriting history.
    \item How many people have heard about rewriting history in Git?
    \item Git does not rewrite history.
    \item What Git does, is write a new history and move a pointer to it.
    \item Old history is still in the database.
    \item If you delete a branch, you're not deleting the work on that branch,
          you're deleting a pointer to it.
    \item Git keeps a log off all of this, so you can go back and find it.
}

\begin{frame}
    \frametitle{What is Git?}
    \centering
    Cryptographically \alert{secure}
\end{frame}
\note[itemize]{
    \item Everything is hashed and addressed by its hash.
    \item Change content, change how you get that content.
    \item Sign tags and commits with PGP.
    \item Git can detect corruption.
}

\begin{frame}
    \frametitle{Git is Popular}
    \centering
    \begin{tikzpicture}
        \begin{axis}[
            title={Google Trends Since First Git Release},
            date coordinates in=x,
            xmin=2005-04-01,
            xmax=2015-08-09,
            ymin=0,
            ymax=100,
            width=\textwidth,
            height=0.8\textheight,
            yticklabels={,,},
            tick style={draw=none},
            xtick={2006-01-01,2007-01-01,2008-01-01,2009-01-01,2010-01-01,2011-01-01,2012-01-01,2013-01-01,2014-01-01,2015-01-01},
            xticklabel=\year,
            xticklabel style={
                font=\scriptsize,
            },
            axis x line*=bottom,
            %hide y axis,
            axis y line*=left,
            axis line style={solarized-base01},
            legend style={
                at={(0.6,1)},
                fill=solarized-base3!40,
                %draw=none,
                draw=solarized-base01,
                font=\scriptsize,
            },
            legend columns=-1,
        ]
            \addplot[solarized-orange] table [col sep=comma,x=Week,y=Git] {google-trends.csv};
            \addplot[solarized-blue] table [col sep=comma,x=Week,y=Subversion] {google-trends.csv};
            \addplot[solarized-violet] table [col sep=comma,x=Week,y=Perforce] {google-trends.csv};
            \legend{Git,Subversion,Perforce}
        \end{axis}
    \end{tikzpicture}
\end{frame}
\note[itemize]{
    \item Dip every Christmas
    \item I think the peak 2012 was due to the release of GitHub for Windows on May 21, 2012
}
