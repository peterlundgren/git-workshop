\section[Section]{What is Git?}

\tikzset{
    box/.style={
        draw=solarized-base01,
        fill=solarized-base3!50,
        %thick,
        drop shadow={
            opacity=0.15,
        },
    },
    box2/.style={
        draw=solarized-base01,
        fill=solarized-base2!50,
        %thick,
        drop shadow={
            opacity=0.15,
        },
    },
    arrow/.style={
        %semithick,
        Latex-,
        draw=gray,
    },
    file/.style={
        shape=document,
        minimum height=4em,
        minimum width=3em,
        draw=solarized-base01,
        fill=solarized-blue!20,
        %thick,
        drop shadow={
            opacity=0.15,
        },
        font=\fontfamily{lmtt}\selectfont\small,
    },
    version/.style={
        shape=rounded rectangle,
        rounded rectangle arc length=90,
        minimum height=1.6em,
        minimum width=2em,
        draw=solarized-base01,
        fill=solarized-green!20,
        %very thick,
        drop shadow={
            opacity=0.15,
        },
        font=\fontfamily{lmtt}\selectfont\small,
    },
}

\newsavebox{\versiondatabase}
\savebox{\versiondatabase}{%
    \begin{tikzpicture}[node distance=1em]
        \pgfdeclarelayer{background}
        \pgfsetlayers{background,main}
        \node (vdb) [] {Version Database};
        \node (v3) [below=of vdb, version] {Version 3};
        \node (v2) [below=of v3, version] {Version 2};
        \node (v1) [below=of v2, version] {Version 1};
        \draw [arrow] (v2) -- (v3);
        \draw [arrow] (v1) -- (v2);
        \begin{pgfonlayer}{background}
            \node (vdbbox) [fit=(vdb)(v3)(v2)(v1),box2] {};
        \end{pgfonlayer}
    \end{tikzpicture}%
}

\newsavebox{\files}
\savebox{\files}{%
    \begin{tikzpicture}[]
        \foreach \i in {1,2,3} {
            \begin{scope}[shift={(.1*\i,.1*\i)}]
                \node () [file] {Files};
            \end{scope}
        }
    \end{tikzpicture}%
}

\begin{frame}
    \frametitle{What is Git?}
    \centering
    \minibox{
        git \textit{noun} \textbackslash'git\textbackslash\\
        \textit{British}\\
        \qquad : a foolish or worthless person
    }
\end{frame}

\begin{frame}
    \frametitle{What is Git?}
    Random three-letter combination that is pronounceable, and not actually
    used by any common UNIX command. The fact that it is a mispronunciation of
    "get" may or may not be relevant.
\end{frame}
\note{From the git README}

\begin{frame}
    \frametitle{What is Git?}
    \centering
    Git is an open source,\\
    distributed version control system\\
    designed for speed and efficiency
\end{frame}
\note{Official tagline of Git}

\begin{frame}
    \frametitle{What is Git?}
    \centering
    Git is an \alert{open source},\\
    distributed version control system\\
    designed for speed and efficiency
\end{frame}
\note[itemize]{
    \item LGPL-2.1
    \item https://github.com/git/git
}

\begin{frame}
    \frametitle{Linus Torvalds}
    \begin{columns}
        \begin{column}{0.38\textwidth}
            \includegraphics[width=1.2\textwidth]{Linus-Torvalds}
        \end{column}
        \begin{column}{0.62\textwidth}
            \chapquote{``I'm an egotistical bastard, and I name all my projects
                    after myself. First 'Linux', now 'Git'"}{Linus Torvalds}
        \end{column}
    \end{columns}
\end{frame}
\note[itemize]{
    \item First release was 7 April 2005 from Linus Torvalds
    \item Kernel hacker mentality. It was written to manage the Linux
          kernel. So, it won't stop you from shooting yourself in the foot.
}

\begin{frame}
    \frametitle{Junio Hamano}
    \centering
    \includegraphics[width=\textwidth,height=0.8\textheight,keepaspectratio]{Junio-Hamano}
\end{frame}
\note[itemize]{
    \item Maintained by Junio Hamano since since 26 July 2005
}

\begin{frame}
    \frametitle{And Many More}
    {
        \fontsize{2.5}{4}\selectfont
        \setlength{\parskip}{3pt}
        \setlength{\parindent}{0pt}
        \setitemize[0]{leftmargin=*,rightmargin=0pt}
        \color{black}
        \setlength{\columnsep}{0pt}
        \begin{multicols}{7}
            \begin{itemize}
                \item[] Junio C Hamano
                \item[] Jeff King
                \item[] Shawn O. Pearce
                \item[] Linus Torvalds
                \item[] Nguyễn Thái Ngọc Duy
                \item[] Johannes Schindelin
                \item[] Michael Haggerty
                \item[] Jonathan Nieder
                \item[] René Scharfe
                \item[] Eric Wong
                \item[] Jakub Narębski
                \item[] Christian Couder
                \item[] Johannes Sixt
                \item[] Felipe Contreras
                \item[] Nicolas Pitre
                \item[] Paul Mackerras
                \item[] Thomas Rast
                \item[] Brandon Casey
                \item[] Ævar Arnfjörð Bjarmason
                \item[] Matthieu Moy
                \item[] Michael J Gruber
                \item[] Simon Hausmann
                \item[] Jiang Xin
                \item[] Petr Baudis
                \item[] Alex Riesen
                \item[] Stefan Beller
                \item[] Ramkumar Ramachandra
                \item[] Elia Pinto
                \item[] Eric Sunshine
                \item[] Johan Herland
                \item[] Miklos Vajna
                \item[] Ramsay Jones
                \item[] Daniel Barkalow
                \item[] J. Bruce Fields
                \item[] SZEDER Gábor
                \item[] David Aguilar
                \item[] John Keeping
                \item[] Pete Wyckoff
                \item[] Elijah Newren
                \item[] Kay Sievers
                \item[] Pierre Habouzit
                \item[] Jens Lehmann
                \item[] Stephen Boyd
                \item[] Tay Ray Chuan
                \item[] Ralf Thielow
                \item[] Paul Tan
                \item[] Alexandre Julliard
                \item[] Karsten Blees
                \item[] Martin von Zweigbergk
                \item[] Pat Thoyts
                \item[] Clemens Buchacher
                \item[] Giuseppe Bilotta
                \item[] Alexander Gavrilov
                \item[] Avery Pennarun
                \item[] Jay Soffian
                \item[] Torsten Bögershausen
                \item[] brian m. carlson
                \item[] Erik Faye-Lund
                \item[] Karthik Nayak
                \item[] Fredrik Kuivinen
                \item[] Nanako Shiraishi
                \item[] Ronnie Sahlberg
                \item[] Jim Meyering
                \item[] Frank Lichtenheld
                \item[] Jon Seymour
                \item[] Steffen Prohaska
                \item[] Brian Gernhardt
                \item[] Gerrit Pape
                \item[] Martin Langhoff
                \item[] Heiko Voigt
                \item[] Mike Hommey
                \item[] David Turner
                \item[] Vasco Almeida
                \item[] Peter Krefting
                \item[] Jonas Fonseca
                \item[] Markus Heidelberg
                \item[] Matthias Lederhofer
                \item[] Bert Wesarg
                \item[] Lars Hjemli
                \item[] Stephan Beyer
                \item[] Michele Ballabio
                \item[] Matthias Urlichs
                \item[] Kirill Smelkov
                \item[] Martin Koegler
                \item[] Nick Hengeveld
                \item[] Christian Stimming
                \item[] Andy Parkins
                \item[] Sergey Vlasov
                \item[] H. Peter Anvin
                \item[] Luben Tuikov
                \item[] Ryan Anderson
                \item[] Charles Bailey
                \item[] Mark Levedahl
                \item[] Sebastian Schuberth
                \item[] Luke Diamand
                \item[] Philip Oakley
                \item[] Thomas Ackermann
                \item[] Trần Ngọc Quân
                \item[] Ben Walton
                \item[] Lars Schneider
                \item[] Pavel Roskin
                \item[] Santi Béjar
                \item[] Adam Spiers
                \item[] Dmitry Potapov
                \item[] Marius Storm-Olsen
                \item[] Sverre Rabbelier
                \item[] Dan McGee
                \item[] Jon Loeliger
                \item[] Sean Estabrooks
                \item[] Sven Verdoolaege
                \item[] W. Trevor King
                \item[] Sam Vilain
                \item[] Björn Gustavsson
                \item[] Carlos Martín Nieto
                \item[] Uwe Kleine-König
                \item[] Aneesh Kumar K.V
                \item[] Lukas Sandström
                \item[] Matthew Ogilvie
                \item[] Han-Wen Nienhuys
                \item[] Michael G. Schwern
                \item[] Theodore Ts'o
                \item[] Wincent Colaiuta
                \item[] David Barr
                \item[] Michał Kiedrowicz
                \item[] Zbigniew Jędrzejewski-Szmek
                \item[] Andreas Ericsson
                \item[] Célestin Matte
                \item[] Patrick Steinhardt
                \item[] Dennis Stosberg
                \item[] Kyle J. McKay
                \item[] Tanay Abhra
                \item[] Alex Henrie
                \item[] Antoine Pelisse
                \item[] David Kastrup
                \item[] Jacob Keller
                \item[] Jean-Noel Avila
                \item[] Ralf Wildenhues
                \item[] Stefano Lattarini
                \item[] Julian Phillips
                \item[] Eric W. Biederman
                \item[] Ilari Liusvaara
                \item[] Martin Waitz
                \item[] Timo Hirvonen
                \item[] Yann Dirson
                \item[] Björn Steinbrink
                \item[] Kevin Ballard
                \item[] Richard Hansen
                \item[] Thomas Gummerer
                \item[] Carlos Rica
                \item[] Kjetil Barvik
                \item[] Kristian Høgsberg
                \item[] Max Kirillov
                \item[] Michael S. Tsirkin
                \item[] Paolo Bonzini
                \item[] Andy Whitcroft
                \item[] Jason Riedy
                \item[] Kevin Bracey
                \item[] Mark Lodato
                \item[] Michael Witten
                \item[] Robert Fitzsimons
                \item[] Robin Rosenberg
                \item[] Tim Henigan
                \item[] Alexander Shopov
                \item[] Dmitry Ivankov
                \item[] Karl Wiberg
                \item[] Peter Eriksen
                \item[] Brad King
                \item[] Josef Weidendorfer
                \item[] Matthias Kestenholz
                \item[] Vitor Antunes
                \item[] Adam Roben
                \item[] Anders Kaseorg
                \item[] Brian Downing
                \item[] Jari Aalto
                \item[] Lee Marlow
            \end{itemize}
        \end{multicols}
    }
\end{frame}
\note[itemize]{
    \item Just a few of Git's contributors
    \item 1430 Contributors listed as of 2016-08-30
}

\begin{frame}
    \frametitle{What is Git?}
    \centering
    Git is an open source,\\
    \alert{distributed} version control system\\
    designed for speed and efficiency
\end{frame}

\begin{frame}
    \frametitle{Centralized Version Control}
    \centering
    \begin{figure}
        \begin{tikzpicture}[node distance=1em]
            \pgfdeclarelayer{background}
            \pgfsetlayers{background,main}
            \node (server) [] {Server};
            \node (vdbbox) [below=of server,inner sep=0] {\usebox{\versiondatabase}};
            \node (client1) [] at (-4,-2) {Client A};
            \node (files1) [below=of client1,inner sep=0] {\usebox{\files}};
            \node (client2) [] at (4,-2) {Client B};
            \node (files2) [below=of client2,inner sep=0] {\usebox{\files}};
            \begin{pgfonlayer}{background}
                \node [fit=(server)(vdbbox),box] {};
                \node [fit=(client1)(files1),box] {};
                \node [fit=(client2)(files2),box] {};
            \end{pgfonlayer}
            \draw [arrow] (vdbbox) -- (files1);
            \draw [arrow] (files1) -- (vdbbox);
            \draw [arrow] (vdbbox) -- (files2);
            \draw [arrow] (files2) -- (vdbbox);
        \end{tikzpicture}
        \caption{Centralized Version Control}
    \end{figure}
\end{frame}
\note[itemize]{
    \item Client, server model
    \item Version database on only one server
    \item Download a snapshot
    \item Send incremental changes to the server
    \item Division of responsibility; some things only server can do, some
          things only client can do.
}

\begin{frame}
    \frametitle{Distributed Version Control}
    \begin{figure}
        \begin{adjustbox}{max totalsize={\textwidth}{\textheight},center}
            \begin{tikzpicture}[node distance=1.4em]
                \pgfdeclarelayer{background}
                \pgfsetlayers{background,main}
                \node (server) [] {Server};
                \node (vdbbox) [below=of server,inner sep=0] {\usebox{\versiondatabase}};
                \node (client1) [] at (-4,0) {Client A};
                \node (files1) [below=of client1,inner sep=0] {\usebox{\files}};
                \node (vdbbox1) [below=of files1,inner sep=0] {\usebox{\versiondatabase}};
                \node (client2) [] at (4,0) {Client B};
                \node (files2) [below=of client2,inner sep=0] {\usebox{\files}};
                \node (vdbbox2) [below=of files2,inner sep=0] {\usebox{\versiondatabase}};
                \begin{pgfonlayer}{background}
                    \node [fit=(server)(vdbbox),box] {};
                    \node [fit=(client1)(files1)(vdbbox1),box] {};
                    \node [fit=(client2)(files2)(vdbbox2),box] {};
                \end{pgfonlayer}
                \draw [arrow] (vdbbox) -- (vdbbox1);
                \draw [arrow] (vdbbox1) -- (vdbbox);
                \draw [arrow] (files1) -- (vdbbox1);
                \draw [arrow] (vdbbox1) -- (files1);
                \draw [arrow] (vdbbox) -- (vdbbox2);
                \draw [arrow] (vdbbox2) -- (vdbbox);
                \draw [arrow] (files2) -- (vdbbox2);
                \draw [arrow] (vdbbox2) -- (files2);
                \draw [arrow] ([yshift=-3em] vdbbox1.east) -- ([yshift=-3em] vdbbox2.west);
                \draw [arrow] ([yshift=-3em] vdbbox2.west) -- ([yshift=-3em] vdbbox1.east);
            \end{tikzpicture}
        \end{adjustbox}
        \caption{Distributed Version Control}
    \end{figure}
\end{frame}
\note[itemize]{
    \item Peer to peer
    \item Version database on every machine
    \item Clients can talk to each other
    \item Download the entire repository
    \item Operate locally, share explicitely
    \item You can have a central server
    \item Servers are only different in that, as an optimization, they
          don't have working copies of the files. The clients are actually
          more featureful than the server.
}

\begin{frame}
    \frametitle{What is Git?}
    \centering
    Git is an open source,\\
    distributed version control system\\
    designed for \alert{speed and efficiency}
\end{frame}

\newcommand{\gitsvnplot}[4]{%
    \begin{tikzpicture}[baseline]
        \ifthenelse{\equal{#1}{true}}{
            \begin{axis}[
                ybar=0,
                bar width=1.2cm,
                x=1.2cm,
                enlarge x limits={abs=0.1cm},
                ymin=0,
                symbolic x coords={#2},
                xtick=data,
                yticklabels={,,},
                tick style={draw=none},
                axis x line*=bottom,
                xticklabel style={
                    font=\small,
                },
                legend style={
                    %at={(1.5,1)},
                    fill=solarized-base3!40,
                    %draw=none,
                    draw=solarized-base01,
                },
                legend columns=-1,
                legend to name=foobar,
                legend entries={Git,Subversion},
                nodes near coords,
                nodes near coords={\pgfmathprintnumber[fixed,precision=2,zerofill=true]{\pgfplotspointmeta}},
                axis line style={solarized-base01},
                axis y line*=left,
                ylabel={Seconds},
                ylabel near ticks,
            ]
        }{
            \begin{axis}[
                ybar=0,
                bar width=1.2cm,
                x=1.2cm,
                enlarge x limits={abs=0.1cm},
                ymin=0,
                symbolic x coords={#2},
                xtick=data,
                yticklabels={,,},
                tick style={draw=none},
                axis x line*=bottom,
                xticklabel style={
                    font=\small,
                },
                legend style={
                    %at={(1.5,1)},
                    fill=solarized-base3!40,
                    %draw=none,
                    draw=solarized-base01,
                },
                legend columns=-1,
                legend to name=foobar,
                legend entries={Git,Subversion},
                nodes near coords,
                nodes near coords={\pgfmathprintnumber[fixed,precision=2,zerofill=true]{\pgfplotspointmeta}},
                scaled y ticks=false,
                axis line style={solarized-base01},
                hide y axis,
            ]
        }
            \addplot[solarized-orange,fill=solarized-orange!20] coordinates {(#2,#3)};
            \addplot[solarized-blue,fill=solarized-blue!20] coordinates {(#2,#4)};
        \end{axis}
    \end{tikzpicture}%
}
\begin{frame}
    \frametitle{Git is Fast}
    \begin{figure}
        \begin{adjustbox}{max totalsize={\textwidth}{.75\textheight},center}
            \begin{tabular}{llllll}
                \gitsvnplot{true}{Commit Files}{0.64}{2.60} &
                \gitsvnplot{false}{Commit Images}{1.53}{24.70} &
                \gitsvnplot{false}{Diff Current}{0.25}{1.09} &
                \gitsvnplot{false}{Diff Recent}{0.25}{3.99} &
                \gitsvnplot{false}{Diff Tags}{1.17}{83.57} &
                \gitsvnplot{false}{Clone}{107.5}{14.0} \\
                \gitsvnplot{true}{Log (50)}{0.01}{0.38} &
                \gitsvnplot{false}{Log (All)}{0.52}{169.20} &
                \gitsvnplot{false}{Log (File)}{0.60}{82.84} &
                \gitsvnplot{false}{Update}{0.90}{2.82} &
                \gitsvnplot{false}{Blame}{1.91}{3.04} &
                \gitsvnplot{false}{Size}{181.0}{132.0} \\
                \multicolumn{6}{r}{\ref{foobar}} \\
            \end{tabular}
        \end{adjustbox}
        \caption{Runtime of Git and Subversion Commands}
    \end{figure}
\end{frame}
\note[itemize]{
    \item Data from Scott Chacon http://git-scm.com/about/small-and-fast
    \item All operations are local except explicit synchronization
    \item No network access needed to:
    \begin{itemize}
        \item Perform a diff
        \item View file history
        \item Commit changes
        \item Merge branches
        \item Switch branches
        \item Checkout another revision
        \item Blame a file
        \item Search for the change that introduced a bug
    \end{itemize}
}

\begin{frame}
    \frametitle{What is Git?}
    \centering
    \alert{Immutable} \\
    (almost) never removes data
\end{frame}
\note[itemize]{
    \item You will hear about rewriting history.
    \item How many people have heard about rewriting history in Git?
    \item Git does not rewrite history.
    \item What Git does, is write a new history and move a pointer to it.
    \item Old history is still in the database.
    \item If you delete a branch, you're not deleting the work on that branch,
          you're deleting a pointer to it.
    \item Git keeps a log off all of this, so you can go back and find it.
}

\begin{frame}
    \frametitle{What is Git?}
    \centering
    Cryptographically \alert{secure}
\end{frame}
\note[itemize]{
    \item Everything is hashed and addressed by its hash.
    \item Change content, change how you get that content.
    \item Sign tags and commits with PGP.
    \item Git can detect corruption.
}

\begin{frame}
    \frametitle{Git is Popular}
    \centering
    % https://www.google.com/trends/explore?date=2005-05-01%202016-08-31&q=%2Fm%2F05vqwg,Subversion,%2Fm%2F08w6d6
    % Download csv in 4 year chunks to get weekly data
    % Adjust scale so that they match on overlapping weeks
    \begin{figure}
        \begin{tikzpicture}
            \begin{axis}[
                date coordinates in=x,
                xmin=2005-04-01,
                xmax=2016-09-01,
                ymin=0,
                ymax=226,
                width=\textwidth,
                height=0.8\textheight,
                yticklabels={,,},
                tick style={draw=none},
                xtick={2006-01-01,2008-01-01,2010-01-01,2012-01-01,2014-01-01,2016-01-01},
                xticklabel=\year,
                xticklabel style={
                    font=\scriptsize,
                },
                axis x line*=bottom,
                %hide y axis,
                axis y line*=left,
                axis line style={solarized-base01},
                legend style={
                    at={(0.6,1)},
                    fill=solarized-base3!40,
                    %draw=none,
                    draw=solarized-base01,
                    font=\scriptsize,
                },
                legend columns=-1,
            ]
                \addplot[solarized-orange] table [col sep=comma,x=Week,y=Git] {google-trends.csv};
                \addplot[solarized-blue] table [col sep=comma,x=Week,y=Subversion] {google-trends.csv};
                \addplot[solarized-violet] table [col sep=comma,x=Week,y=Perforce] {google-trends.csv};
                \legend{Git,Subversion,Perforce}
            \end{axis}
        \end{tikzpicture}
        \caption{Google Trends Since First Git Release}
    \end{figure}
\end{frame}
\note[itemize]{
    \item Dip every Christmas
}
