% Funny bug http://tex.stackexchange.com/questions/232168/normal-text-is-invisible-when-using-beamer-with-notes-and-xelatex
\def\pgfsysdriver{pgfsys-dvipdfm.def}
\documentclass{beamer}
\title{Git Workshop}
\author{Peter Lundgren}

\usepackage{pgfpages}
\usepackage{ifthen}
\usepackage{pgfplots}
\pgfplotsset{compat=1.11}
\usepgfplotslibrary{dateplot}
\usepackage{adjustbox}
\usepackage{minibox}
\usepackage{subcaption}
\captionsetup{compatibility=false}
\usepackage{gitdags}
\usepackage{xcolor-solarized}
\usepackage{tikz}
\usetikzlibrary{fit,arrows,shadows}
\usepackage{hyperref}
\usepackage{multicol}
\usepackage{graphics}
\graphicspath{ {images/} }
\usepackage{enumitem}
% Don't let enumitem redefine beamer's template
\setitemize{label=\usebeamerfont*{itemize item}%
  \usebeamercolor[fg]{itemize item}
  \usebeamertemplate{itemize item}}

\makeatletter
% Courtesy of Section 102.5.3 of the Tikz & PGF Manual
\pgfdeclareshape{document}{
    \inheritsavedanchors[from=rectangle] % this is nearly a rectangle
    \inheritanchorborder[from=rectangle]
    \inheritanchor[from=rectangle]{center}
    \inheritanchor[from=rectangle]{north}
    \inheritanchor[from=rectangle]{south}
    \inheritanchor[from=rectangle]{west}
    \inheritanchor[from=rectangle]{east}
    % ... and possibly more
    \backgroundpath{% this is new
        % store lower right in xa/ya and upper right in xb/yb
        \southwest \pgf@xa=\pgf@x \pgf@ya=\pgf@y
        \northeast \pgf@xb=\pgf@x \pgf@yb=\pgf@y
        % compute corner of ‘‘flipped page’’
        \pgf@xc=\pgf@xb \advance\pgf@xc by-7.5pt % this should be a parameter
        \pgf@yc=\pgf@yb \advance\pgf@yc by-7.5pt
        % construct main path
        \pgfpathmoveto{\pgfpoint{\pgf@xa}{\pgf@ya}}
        \pgfpathlineto{\pgfpoint{\pgf@xa}{\pgf@yb}}
        \pgfpathlineto{\pgfpoint{\pgf@xc}{\pgf@yb}}
        \pgfpathlineto{\pgfpoint{\pgf@xb}{\pgf@yc}}
        \pgfpathlineto{\pgfpoint{\pgf@xb}{\pgf@ya}}
        \pgfpathclose
        % add little corner
        \pgfpathmoveto{\pgfpoint{\pgf@xc}{\pgf@yb}}
        \pgfpathlineto{\pgfpoint{\pgf@xc}{\pgf@yc}}
        \pgfpathlineto{\pgfpoint{\pgf@xb}{\pgf@yc}}
        \pgfpathlineto{\pgfpoint{\pgf@xc}{\pgf@yc}}
    }
}
\makeatother

\newcommand{\chapquote}[2]{
    \begin{quotation} {\itshape#1} \end{quotation}
    \begin{flushright} - #2 \end{flushright}
}

\AtBeginSection[]{
    \begin{frame}
    \frametitle{Table of Contents}
    \tableofcontents[currentsection]
    \end{frame}
}

\setbeamercolor{normal text}{fg=black}
\setbeamercolor{structure}{fg=solarized-blue,bg=solarized-base2!50}
\setbeamercolor{alerted text}{fg=solarized-orange}
\setbeamerfont{alerted text}{series=\bfseries}
\setbeamercolor{background canvas}{bg=solarized-base3!40}
\setbeamertemplate{background}{
    \tikz[overlay,remember picture]
    \node[at=(current page.south east),anchor=south east,inner sep=0pt] {
    \includegraphics[width=6em]{Git-Icon-base3.png}};
}
\setbeameroption{show notes on second screen=right}
\setbeamertemplate{note page}[plain]

\begin{document}
    \begin{frame}
        \titlepage
    \end{frame}
    \note[itemize]{
        \item Introductions
        \item We've got a lot to talk about today
        \item We've also got a lot of time
        \item Ask questions
        \item Tell me if I'm going too fast
    }

    \begin{frame}
        \frametitle{More Information}
        \centering
        This presentation is avaliable at
        \url{https://github.com/peterlundgren/git-workshop}

        \bigskip
        Download Git at
        \url{https://git-scm.com}

        \bigskip
        Learn more about Git at
        \url{https://progit.org}
    \end{frame}

    \section[Section]{What is Git?}

\tikzset{
    box/.style={
        draw=solarized-base01,
        fill=solarized-base3!50,
        %thick,
        drop shadow={
            opacity=0.15,
        },
    },
    box2/.style={
        draw=solarized-base01,
        fill=solarized-base2!50,
        %thick,
        drop shadow={
            opacity=0.15,
        },
    },
    arrow/.style={
        %semithick,
        Latex-,
        draw=gray,
    },
    file/.style={
        shape=document,
        minimum height=4em,
        minimum width=3em,
        draw=solarized-base01,
        fill=solarized-blue!20,
        %thick,
        drop shadow={
            opacity=0.15,
        },
        font=\fontfamily{lmtt}\selectfont\small,
    },
    version/.style={
        shape=rounded rectangle,
        rounded rectangle arc length=90,
        minimum height=1.6em,
        minimum width=2em,
        draw=solarized-base01,
        fill=solarized-green!20,
        %very thick,
        drop shadow={
            opacity=0.15,
        },
        font=\fontfamily{lmtt}\selectfont\small,
    },
}

\newsavebox{\versiondatabase}
\savebox{\versiondatabase}{%
    \begin{tikzpicture}[node distance=1em]
        \pgfdeclarelayer{background}
        \pgfsetlayers{background,main}
        \node (vdb) [] {Version Database};
        \node (v3) [below=of vdb, version] {Version 3};
        \node (v2) [below=of v3, version] {Version 2};
        \node (v1) [below=of v2, version] {Version 1};
        \draw [arrow] (v2) -- (v3);
        \draw [arrow] (v1) -- (v2);
        \begin{pgfonlayer}{background}
            \node (vdbbox) [fit=(vdb)(v3)(v2)(v1),box2] {};
        \end{pgfonlayer}
    \end{tikzpicture}%
}

\newsavebox{\files}
\savebox{\files}{%
    \begin{tikzpicture}[]
        \foreach \i in {1,2,3} {
            \begin{scope}[shift={(.1*\i,.1*\i)}]
                \node () [file] {Files};
            \end{scope}
        }
    \end{tikzpicture}%
}

\begin{frame}
    \frametitle{What is Git?}
    \centering
    \minibox{
        git \textit{noun} \textbackslash'git\textbackslash\\
        \textit{British}\\
        \qquad : a foolish or worthless person
    }
\end{frame}

\begin{frame}
    \frametitle{What is Git?}
    \chapquote{``I'm an egotistical bastard, and I name all my projects after
               myself. First 'Linux', now 'Git'"}{Linux Torvalds}
\end{frame}

\begin{frame}
    \frametitle{What is Git?}
    Random three-letter combination that is pronounceable, and not actually
    used by any common UNIX command. The fact that it is a mispronunciation of
    "get" may or may not be relevant.
\end{frame}

\begin{frame}
    \frametitle{What is Git?}
    \centering
    Git is an open source,\\
    distributed version control system\\
    designed for speed and efficiency
\end{frame}
\note{Official tagline of Git}

\begin{frame}
    \frametitle{What is Git?}
    \centering
    Git is an \alert{open source},\\
    distributed version control system\\
    designed for speed and efficiency
\end{frame}
\note[itemize]{
    \item LGPL-2.1
    \item https://github.com/git/git
    \item First release was 7 April 2005 from Linus Torvalds
    \item Maintained by Junio Hamano since since 26 July 2005
    \item Kernel hacker mentality. It was written to manage the Linux
          kernel. So, it won't stop you from shooting yourself in the foot.
}

\begin{frame}
    \frametitle{What is Git?}
    \centering
    Git is an open source,\\
    \alert{distributed} version control system\\
    designed for speed and efficiency
\end{frame}

\begin{frame}
    \frametitle{Centralized Version Control}
    \centering
    \begin{tikzpicture}[node distance=1em]
        \pgfdeclarelayer{background}
        \pgfsetlayers{background,main}
        \node (server) [] {Server};
        \node (vdbbox) [below=of server,inner sep=0] {\usebox{\versiondatabase}};
        \node (client1) [] at (-4,-2) {Client A};
        \node (files1) [below=of client1,inner sep=0] {\usebox{\files}};
        \node (client2) [] at (4,-2) {Client B};
        \node (files2) [below=of client2,inner sep=0] {\usebox{\files}};
        \begin{pgfonlayer}{background}
            \node [fit=(server)(vdbbox),box] {};
            \node [fit=(client1)(files1),box] {};
            \node [fit=(client2)(files2),box] {};
        \end{pgfonlayer}
        \draw [arrow] (vdbbox) -- (files1);
        \draw [arrow] (files1) -- (vdbbox);
        \draw [arrow] (vdbbox) -- (files2);
        \draw [arrow] (files2) -- (vdbbox);
    \end{tikzpicture}
\end{frame}
\note[itemize]{
    \item Client, server model
    \item Version database on only one server
    \item Download a snapshot
    \item Send incremental changes to the server
    \item Division of responsibility; some things only server can do, some
          things only client can do.
}

\begin{frame}
    \frametitle{Distributed Version Control}
    \begin{adjustbox}{max totalsize={\textwidth}{\textheight},center}
        \begin{tikzpicture}[node distance=1.4em]
            \pgfdeclarelayer{background}
            \pgfsetlayers{background,main}
            \node (server) [] {Server};
            \node (vdbbox) [below=of server,inner sep=0] {\usebox{\versiondatabase}};
            \node (client1) [] at (-4,0) {Client A};
            \node (files1) [below=of client1,inner sep=0] {\usebox{\files}};
            \node (vdbbox1) [below=of files1,inner sep=0] {\usebox{\versiondatabase}};
            \node (client2) [] at (4,0) {Client B};
            \node (files2) [below=of client2,inner sep=0] {\usebox{\files}};
            \node (vdbbox2) [below=of files2,inner sep=0] {\usebox{\versiondatabase}};
            \begin{pgfonlayer}{background}
                \node [fit=(server)(vdbbox),box] {};
                \node [fit=(client1)(files1)(vdbbox1),box] {};
                \node [fit=(client2)(files2)(vdbbox2),box] {};
            \end{pgfonlayer}
            \draw [arrow] (vdbbox) -- (vdbbox1);
            \draw [arrow] (vdbbox1) -- (vdbbox);
            \draw [arrow] (files1) -- (vdbbox1);
            \draw [arrow] (vdbbox1) -- (files1);
            \draw [arrow] (vdbbox) -- (vdbbox2);
            \draw [arrow] (vdbbox2) -- (vdbbox);
            \draw [arrow] (files2) -- (vdbbox2);
            \draw [arrow] (vdbbox2) -- (files2);
            \draw [arrow] ([yshift=-3em] vdbbox1.east) -- ([yshift=-3em] vdbbox2.west);
            \draw [arrow] ([yshift=-3em] vdbbox2.west) -- ([yshift=-3em] vdbbox1.east);
        \end{tikzpicture}
    \end{adjustbox}
\end{frame}
\note[itemize]{
    \item Peer to peer
    \item Version database on every machine
    \item Clients can talk to each other
    \item Download the entire repository
    \item Operate locally, share explicitely
    \item You can have a central server
    \item Servers are only different in that, as an optimization, they
          don't have working copies of the files. The clients are actually
          more featureful than the server.
}

\begin{frame}
    \frametitle{What is Git?}
    \centering
    Git is an open source,\\
    distributed version control system\\
    designed for \alert{speed and efficiency}
\end{frame}

\newcommand{\gitsvnplot}[4]{%
    \begin{tikzpicture}[baseline]
        \ifthenelse{\equal{#1}{true}}{
            \begin{axis}[
                ybar=0,
                bar width=1.2cm,
                x=1.2cm,
                enlarge x limits={abs=0.1cm},
                ymin=0,
                symbolic x coords={#2},
                xtick=data,
                yticklabels={,,},
                tick style={draw=none},
                axis x line*=bottom,
                xticklabel style={
                    font=\small,
                },
                legend style={
                    %at={(1.5,1)},
                    fill=solarized-base3!40,
                    %draw=none,
                    draw=solarized-base01,
                },
                legend columns=-1,
                legend to name=foobar,
                legend entries={Git,Subversion},
                nodes near coords,
                nodes near coords={\pgfmathprintnumber[fixed,precision=2,zerofill=true]{\pgfplotspointmeta}},
                axis line style={solarized-base01},
                axis y line*=left,
                ylabel={Seconds},
                ylabel near ticks,
            ]
        }{
            \begin{axis}[
                ybar=0,
                bar width=1.2cm,
                x=1.2cm,
                enlarge x limits={abs=0.1cm},
                ymin=0,
                symbolic x coords={#2},
                xtick=data,
                yticklabels={,,},
                tick style={draw=none},
                axis x line*=bottom,
                xticklabel style={
                    font=\small,
                },
                legend style={
                    %at={(1.5,1)},
                    fill=solarized-base3!40,
                    %draw=none,
                    draw=solarized-base01,
                },
                legend columns=-1,
                legend to name=foobar,
                legend entries={Git,Subversion},
                nodes near coords,
                nodes near coords={\pgfmathprintnumber[fixed,precision=2,zerofill=true]{\pgfplotspointmeta}},
                scaled y ticks=false,
                axis line style={solarized-base01},
                hide y axis,
            ]
        }
            \addplot[solarized-orange,fill=solarized-orange!20] coordinates {(#2,#3)};
            \addplot[solarized-blue,fill=solarized-blue!20] coordinates {(#2,#4)};
        \end{axis}
    \end{tikzpicture}%
}
\begin{frame}
    \frametitle{Git is Fast}
    \begin{adjustbox}{max totalsize={\textwidth}{.8\textheight},center}
        \begin{tabular}{llllll}
            \gitsvnplot{true}{Commit Files}{0.64}{2.60} &
            \gitsvnplot{false}{Commit Images}{1.53}{24.70} &
            \gitsvnplot{false}{Diff Current}{0.25}{1.09} &
            \gitsvnplot{false}{Diff Recent}{0.25}{3.99} &
            \gitsvnplot{false}{Diff Tags}{1.17}{83.57} &
            \gitsvnplot{false}{Clone}{107.5}{14.0} \\
            \gitsvnplot{true}{Log (50)}{0.01}{0.38} &
            \gitsvnplot{false}{Log (All)}{0.52}{169.20} &
            \gitsvnplot{false}{Log (File)}{0.60}{82.84} &
            \gitsvnplot{false}{Update}{0.90}{2.82} &
            \gitsvnplot{false}{Blame}{1.91}{3.04} &
            \gitsvnplot{false}{Size}{181.0}{132.0} \\
            \multicolumn{6}{r}{\ref{foobar}} \\
        \end{tabular}
    \end{adjustbox}
\end{frame}
\note[itemize]{
    \item Data from Scott Chacon http://git-scm.com/about/small-and-fast
    \item All operations are local except explicit synchronization
    \item No network access needed to:
    \begin{itemize}
        \item Perform a diff
        \item View file history
        \item Commit changes
        \item Merge branches
        \item Switch branches
        \item Checkout another revision
        \item Blame a file
        \item Search for the change that introduced a bug
    \end{itemize}
}

\begin{frame}
    \frametitle{What is Git?}
    \centering
    \alert{Immutable} \\
    (almost) never removes data
\end{frame}
\note[itemize]{
    \item You will hear about rewriting history.
    \item How many people have heard about rewriting history in Git?
    \item Git does not rewrite history.
    \item What Git does, is write a new history and move a pointer to it.
    \item Old history is still in the database.
    \item If you delete a branch, you're not deleting the work on that branch,
          you're deleting a pointer to it.
    \item Git keeps a log off all of this, so you can go back and find it.
}

\begin{frame}
    \frametitle{What is Git?}
    \centering
    Cryptographically \alert{secure}
\end{frame}
\note[itemize]{
    \item Everything is hashed and addressed by its hash.
    \item Change content, change how you get that content.
    \item Sign tags and commits with PGP.
    \item Git can detect corruption.
}

\begin{frame}
    \frametitle{Git is Popular}
    \centering
    \begin{tikzpicture}
        \begin{axis}[
            title={Google Trends Since First Git Release},
            date coordinates in=x,
            xmin=2005-04-01,
            xmax=2015-08-09,
            ymin=0,
            ymax=100,
            width=\textwidth,
            height=0.8\textheight,
            yticklabels={,,},
            tick style={draw=none},
            xtick={2006-01-01,2007-01-01,2008-01-01,2009-01-01,2010-01-01,2011-01-01,2012-01-01,2013-01-01,2014-01-01,2015-01-01},
            xticklabel=\year,
            xticklabel style={
                font=\scriptsize,
            },
            axis x line*=bottom,
            %hide y axis,
            axis y line*=left,
            axis line style={solarized-base01},
            legend style={
                at={(0.6,1)},
                fill=solarized-base3!40,
                %draw=none,
                draw=solarized-base01,
                font=\scriptsize,
            },
            legend columns=-1,
        ]
            \addplot[solarized-orange] table [col sep=comma,x=Week,y=Git] {google-trends.csv};
            \addplot[solarized-blue] table [col sep=comma,x=Week,y=Subversion] {google-trends.csv};
            \addplot[solarized-violet] table [col sep=comma,x=Week,y=Perforce] {google-trends.csv};
            \legend{Git,Subversion,Perforce}
        \end{axis}
    \end{tikzpicture}
\end{frame}
\note[itemize]{
    \item Dip every Christmas
    \item I think the peak 2012 was due to the release of GitHub for Windows on May 21, 2012
}

    \section[Section]{Objectives}

\begin{frame}
\frametitle{Objectives}
\begin{itemize}
    \item Understand how Git works and how to apply that to day to day
          development
    \item Learn the basic 12 everyday commands
    \item Know how to undo mistakes
    \item Learn how to use Git to collaborate
    \item Learn how to find help
\end{itemize}
\end{frame}

\begin{frame}
\frametitle{12 Everyday Commands}
\begin{multicols}{3}
    \begin{itemize}
        \setlength\itemsep{3em}
        \item add
        \item branch
        \item checkout
        \item commit
        \item diff
        \item fetch
        \item help
        \item log
        \item merge
        \item push
        \item rebase
        \item status
    \end{itemize}
\end{multicols}
\end{frame}
\note[itemize]{
    \item Git has 160 subcommands in 2.9.3
    \item I'll cover about 20 of them
    \item These are the 12 you'll use daily
}

\begin{frame}
\frametitle{Finding Help}
\alert{Demo 01}: \texttt{git help}
\end{frame}
\note[itemize]{
    \item I'll cover one of them right now.
    \item Demo 01: git help
}

\begin{frame}
\frametitle{12 Everyday Commands}
\begin{multicols}{3}
    \begin{itemize}
        \setlength\itemsep{3em}
        \item add
        \item branch
        \item checkout
        \item commit
        \item diff
        \item fetch
        \item \alert{help}
        \item log
        \item merge
        \item push
        \item rebase
        \item status
    \end{itemize}
\end{multicols}
\end{frame}
\note[itemize]{
    \item One down, 11 to go.
}

\begin{frame}
\frametitle{Learn 4 Ways}
\begin{itemize}
    \setlength\itemsep{3em}
    \item Conceptual
    \item Commands
    \item Implementation
    \item Try It
\end{itemize}
\end{frame}
\note[itemize]{
    \item Conceptual - Computer science lecture. Diagrams of directed acyclic
          graphs and reachability. I'll lecture. We'll watch some lectures from
          Scott Chacon (Author of "Pro Git", CIO GitHub).
    \item Commands - Practical. How to use common commands.
    \item Implementation - How Git works under the hood.
    \item Try It - Practice!
}

    \section[Section]{Your First Repository}

\begin{frame}
    \frametitle{Your First Repository}
    \alert{Demo 2}: \texttt{git init}
\end{frame}
\note[itemize]{
    \item \alert{Demo 2}: \texttt{git init}
}

    \section[Section]{Three Stage Thinking}

\begin{frame}
    \frametitle{Three Stage Thinking}
    \begin{itemize}
        \setlength\itemsep{3em}
        \item Edit
        \item Add
        \item Commit
    \end{itemize}
\end{frame}

\begin{frame}
    \frametitle{Three Stage Thinking}
    \alert{Demo 3}: Three Stage Thinking
\end{frame}
\note[itemize]{
    \item \alert{Demo 3}: Three Stage Thinking
}

\begin{frame}
    \frametitle{Lesson 1}
    \alert{Lesson 1}: Three Stage Thinking
\end{frame}
\note[itemize]{
    \item \alert{Lesson 1}: Three Stage Thinking
}

\begin{frame}
    \frametitle{Commit Messages}
    The Technical Bits
    \begin{itemize}
        \item Short (aim for 50 characters or less) summary
        \item Followed by a blank line
        \item Body wrapped to 72 characters
    \end{itemize}
\end{frame}
\note[itemize]{
    \item \alert{Webpage}: \url{http://tbaggery.com/2008/04/19/a-note-about-git-commit-messages.html}
    \item Summary line must be separated by a blank space or many tools get a
          little confused
    \item Some views truncate the summary line; soft 50; hard 72
    \item Hard wrap body to 72 characters
    \item \texttt{git log}, \texttt{git format-patch}, etc do not wrap message
}

\begin{frame}
    \frametitle{Commit Messages}
    The Conventional Bits
    \begin{itemize}
        \item Make your commits atomic
        \item Justify your changes; write detailed messages
        \item Write is the imperative: "Fix bug" and not "Fixed bug" or
              "Fixes bug"
        \item Present tense for current commit
        \item Past tense for earlier commits
        \item Future tense for later commits
        \item No period on subject line
        \item Meta-data at the bottom
    \end{itemize}
\end{frame}
\note[itemize]{
    \item These are common expectations
    \item Like most social conventions, will be used to judge you more so than
          that they are technically superior
    \item I include them here so that you can understand and fit in
    \item Atomic: words like and, also, consider splitting commit
    \item Justify: open-source mailing list mentality; what, why, how
    \item Imperative style dates back to, at least, GNU changelogs
    \item Meta-data at the bottom: signed-of-by, change-id, issue tracker
    \item Look at a linux kernel log
}

\begin{frame}
    \frametitle{12 Everyday Commands}
    \begin{multicols}{3}
        \begin{itemize}
            \setlength\itemsep{3em}
            \item \alert{add}
            \item branch
            \item checkout
            \item \alert{commit}
            \item \alert{diff}
            \item fetch
            \item \alert{help}
            \item \alert{log}
            \item merge
            \item push
            \item rebase
            \item \alert{status}
        \end{itemize}
    \end{multicols}
\end{frame}
\note[itemize]{
    \item You've already seen these
}

    \section[Section]{Trees, Hashes, and Blobs}

\begin{frame}
    \frametitle{Trees, Hashes, and Blobs}
    \alert{oh My!}
\end{frame}
\note[itemize]{
    \item \alert{Video}: \url{http://youtu.be/ZDR433b0HJY?t=13m17s} - 0:21:02
    \item By Scott Chacon (Author of "Pro Git", CIO GitHub)
}

\begin{frame}
    \frametitle{Trees, Hashes, and Blobs}
    \alert{Demo 4}: Trees, Hashes, and Blobs
\end{frame}
\note[itemize]{
    \item \alert{Demo 4}: Trees, Hashes, and Blobs
}

\end{document}
