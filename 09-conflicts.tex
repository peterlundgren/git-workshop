\section[Section]{Conflicts}

\begin{frame}
    \frametitle{Managing Conflicts}
    \begin{itemize}
        \item \texttt{git merge}
        \item \texttt{git rebase}
        \item \texttt{git stash pop}
    \end{itemize}
\end{frame}
\note[itemize]{
    \item Merge, rebase, and stash pop can result in conflicts
}

\begin{frame}[fragile]
    \frametitle{Merge Conflict}
    \small
    \begin{minted}[bgcolor=solarized-base2!50,frame=single,framesep=3pt]{console}
$ git merge bugfixes
Auto-merging hello.c
CONFLICT (content): Merge conflict in hello.c
Automatic merge failed; fix conflicts and then commit
the result.
$ git status
# On branch master
# You have unmerged paths.
#   (fix conflicts and run "git commit")
#   (use "git merge --abort" to abort the merge)
#
# Unmerged paths:
#   (use "git add <file>..." to mark resolution)
#
#       both modified:   hello.c
#
no changes added to commit (use "git add" and/or
"git commit -a")
    \end{minted}
\end{frame}
\note[itemize]{
    \item If a merge conflicts, it will stop before creating the merge commit
    \item Git status will remind you that you have conflicts (unmerged paths).
    \item Either abort or resolve the changes
}

\begin{frame}[fragile]
    \frametitle{Merge Conflict Markers}
    \small
    \begin{minted}[bgcolor=solarized-base2!50,frame=single,framesep=3pt]{c}
#include <stdio.h>

int main(void)
{
<<<<<<< HEAD
    printf("Hello Class");
=======
    printf("Hello World\n");
    return 0;
>>>>>>> bugfixes
}
    \end{minted}
\end{frame}
\note[itemize]{
    \item Traditional conflict markers.
    \item Same as in svn.
}


\begin{frame}[fragile]
    \frametitle{Merge Conflict Markers}
    \small
    \begin{minted}[bgcolor=solarized-base2!50,frame=single,framesep=3pt]{console}
$ git config --global merge.conflictstyle diff3
    \end{minted}
    \begin{minted}[bgcolor=solarized-base2!50,frame=single,framesep=3pt]{c}
#include <stdio.h>

int main(void)
{
<<<<<<< HEAD
    printf("Hello Class");
||||||| merged common ancestors
    printf("Hello World");
=======
    printf("Hello World\n");
    return 0;
>>>>>>> bugfixes
}
    \end{minted}
\end{frame}
\note[itemize]{
    \item Alternatively, the diff3 style, which I highly recommend, adds the
          merge base in the middle
}

\begin{frame}[fragile]
    \frametitle{Merge Tools}
    \begin{minted}[bgcolor=solarized-base2!50,frame=single,framesep=3pt]{console}
$ git mergetool --tool=<tool>
    \end{minted}
    \begin{multicols}{3}
    \begin{itemize}
        \item araxis
        \item bc
        \item bc3
        \item codecompare
        \item deltawalker
        \item diffmerge
        \item diffuse
        \item ecmerge
        \item emerge
        \item examdiff
        \item gvimdiff
        \item gvimdiff2
        \item gvimdiff3
        \item kdiff3
        \item meld
        \item opendiff
        \item p4merge
        \item tkdiff
        \item tortoisemerge
        \item vimdiff
        \item vimdiff2
        \item vimdiff3
        \item winmerge
    \end{itemize}
    \end{multicols}
\end{frame}
\note[itemize]{
    \item If you'd like to use a merge tool,
    \item these are all supported out of the box
    \item Can use others with a little bit of configuration to tell git how to
          launch them
}

\begin{frame}
    \frametitle{Merge Conflicts}
    Git stops before creating merge commit. Either:
    \medskip
    \begin{enumerate}[1.]
        \item Abort with \texttt{git merge --abort}
    \end{enumerate}
    \medskip
    or
    \medskip
    \begin{enumerate}[1.]
        \item Resolve the conflicts
        \item Mark files resolved with \texttt{git add <file>}
        \item Finish the merge with \texttt{git commit}
    \end{enumerate}
\end{frame}
\note[itemize]{
    \item Easier said than done...
    \item Conflicts are recorded in merge commit message
}

\begin{frame}
    \frametitle{Lesson 6}
    \alert{Lesson 6}: Merge Conflicts
\end{frame}
\note[itemize]{
    \item ... so let's do it
    \item \alert{Lesson 6}: Merge Conflicts
}
