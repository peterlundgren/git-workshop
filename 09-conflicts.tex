\section[Section]{Conflicts}

\begin{frame}
    \frametitle{Managing Conflicts}
    \begin{itemize}
        \item \texttt{git merge}
        \item \texttt{git rebase}
        \item \texttt{git stash pop}
    \end{itemize}
\end{frame}
\note[itemize]{
    \item Merge, rebase, and stash pop can result in conflicts
}

\begin{frame}[fragile]
    \frametitle{Merge Conflict}
    \small
    \begin{minted}[bgcolor=solarized-base2!50,frame=single,framesep=3pt]{console}
$ git merge bugfixes
Auto-merging hello.c
CONFLICT (content): Merge conflict in hello.c
Automatic merge failed; fix conflicts and then commit
the result.
$ git status
# On branch master
# You have unmerged paths.
#   (fix conflicts and run "git commit")
#   (use "git merge --abort" to abort the merge)
#
# Unmerged paths:
#   (use "git add <file>..." to mark resolution)
#
#       both modified:   hello.c
#
no changes added to commit (use "git add" and/or
"git commit -a")
    \end{minted}
\end{frame}
\note[itemize]{
    \item If a merge conflicts, it will stop before creating the merge commit
    \item \texttt{git status} will remind you that you have conflicts (unmerged paths).
    \item Either abort or resolve the changes
}

\begin{frame}[fragile]
    \frametitle{Merge Conflict Markers}
    \small
    \begin{minted}[bgcolor=solarized-base2!50,frame=single,framesep=3pt]{c}
#include <stdio.h>

int main(void)
{
<<<<<<< HEAD
    printf("Hello Class");
=======
    printf("Hello World\n");
    return 0;
>>>>>>> bugfixes
}
    \end{minted}
\end{frame}
\note[itemize]{
    \item Traditional conflict markers.
    \item Same as in svn.
}


\begin{frame}[fragile]
    \frametitle{Merge Conflict Markers}
    \small
    \begin{minted}[bgcolor=solarized-base2!50,frame=single,framesep=3pt]{console}
$ git config --global merge.conflictstyle diff3
    \end{minted}
    \begin{minted}[bgcolor=solarized-base2!50,frame=single,framesep=3pt]{c}
#include <stdio.h>

int main(void)
{
<<<<<<< HEAD
    printf("Hello Class");
||||||| merged common ancestors
    printf("Hello World");
=======
    printf("Hello World\n");
    return 0;
>>>>>>> bugfixes
}
    \end{minted}
\end{frame}
\note[itemize]{
    \item Alternatively, the diff3 style, which I highly recommend, adds the
          merge base in the middle
}

\begin{frame}[fragile]
    \frametitle{Merge Tools}
    \begin{minted}[bgcolor=solarized-base2!50,frame=single,framesep=3pt]{console}
$ git mergetool --tool=<tool>
    \end{minted}
    \begin{multicols}{3}
    \begin{itemize}
        \item araxis
        \item bc
        \item bc3
        \item codecompare
        \item deltawalker
        \item diffmerge
        \item diffuse
        \item ecmerge
        \item emerge
        \item examdiff
        \item gvimdiff
        \item gvimdiff2
        \item gvimdiff3
        \item kdiff3
        \item meld
        \item opendiff
        \item p4merge
        \item tkdiff
        \item tortoisemerge
        \item vimdiff
        \item vimdiff2
        \item vimdiff3
        \item winmerge
    \end{itemize}
    \end{multicols}
\end{frame}
\note[itemize]{
    \item If you'd like to use a merge tool,
    \item these are all supported out of the box
    \item Can use others with a little bit of configuration to tell git how to
          launch them
}

\begin{frame}
    \frametitle{Merge Conflicts}
    Git stops before creating merge commit. Either:
    \medskip
    \begin{enumerate}[1.]
        \item Abort with \texttt{git merge --abort}
    \end{enumerate}
    \medskip
    or
    \medskip
    \begin{enumerate}[1.]
        \item Resolve the conflicts
        \item Mark files resolved with \texttt{git add <file>}
        \item Finish the merge with \texttt{git commit}
    \end{enumerate}
\end{frame}
\note[itemize]{
    \item Easier said than done...
    \item Conflicts are recorded in merge commit message
}

\begin{frame}
    \frametitle{Lesson 6}
    \alert{Lesson 6}: Merge Conflicts
\end{frame}
\note[itemize]{
    \item ... so let's do it
    \item \alert{Lesson 6}: Merge Conflicts
}

\begin{frame}[fragile]
    \frametitle{Rebase Conflict}
    \small
    \begin{minted}[bgcolor=solarized-base2!50,frame=single,framesep=3pt]{console}
$ git rebase master
...
CONFLICT (content): Merge conflict in hello.c
error: Failed to merge in the changes.
Patch failed at 0002 Print newline
The copy of the patch that failed is found in:
.git/rebase-apply/patch

When you have resolved this problem, run
"git rebase --continue".
If you prefer to skip this patch, run
"git rebase --skip" instead.
To check out the original branch and stop rebasing, run
"git rebase --abort".
    \end{minted}
\end{frame}
\note[itemize]{
    \item Rebase applies each commit one at a time
    \item If a rebase conflicts, it will stop before creating the commit at
          whichever commit in the rebase conflicts
}

\begin{frame}[fragile]
    \frametitle{Rebase Conflict}
    \footnotesize
    \begin{minted}[bgcolor=solarized-base2!50,frame=single,framesep=3pt]{console}
$ git status
# rebase in progress; onto 71f26c3
# You are currently rebasing branch 'bugfixes' on '71f26c3'.
#   (fix conflicts and then run "git rebase --continue")
#   (use "git rebase --skip" to skip this patch)
#   (use "git rebase --abort" to check out the original branch)
#
# Unmerged paths:
#   (use "git reset HEAD <file>..." to unstage)
#   (use "git add <file>..." to mark resolution)
#
#       both modified:   hello.c
#
no changes added to commit (use "git add" and/or
"git commit -a")
    \end{minted}
\end{frame}
\note[itemize]{
    \item \texttt{git status} will remind you that you have a rebase in
          progress with conflicts (unmerged paths).
    \item Either abort, skip, or resolve the changes
    \item The skip option is new
    \item Skip will not attempt to apply that change as if you had ommitted it
          with a \texttt{git rebase -i}
}

\begin{frame}
    \frametitle{Rebase Conflicts}
    Git stops before each conflicting commit. Either:
    \medskip
    \begin{enumerate}[1.]
        \item Abort with \texttt{git rebase --abort}
    \end{enumerate}
    \medskip
    or
    \medskip
    \begin{enumerate}[1.]
        \item Skip one patch with \texttt{git rebase --skip}
    \end{enumerate}
    \medskip
    or
    \medskip
    \begin{enumerate}[1.]
        \item Resolve the conflicts
        \item Mark files resolved with \texttt{git add <file>}
        \item Continue the rebase with \texttt{git rebase --continue}
    \end{enumerate}
\end{frame}
\note[itemize]{
    \item Similar process as a merge conflict but with the added option to skip
    \item Note that the command to continue is no longer \texttt{git commit}
}

\begin{frame}
    \frametitle{Lesson 7}
    \alert{Lesson 7}: Rebase Conflicts
\end{frame}
\note[itemize]{
    \item \alert{Lesson 7}: Rebase Conflicts
    \item After lesson:
          How was this different than resolving the merge conflict?
    \item You can get similar one commit at a time merge conflict resolution
          with \texttt{git rerere}
}
