\section[Section]{Recovering Your Work}

\begin{frame}
    \frametitle{Reachability and Garbage Collection}
    \begin{figure}
        \centering
        \begin{tikzpicture}
            \gitDAG[grow right sep=2em]{
                A -- {{B, C -- D} -- H, E -- F -- G} -- I
            };
        \end{tikzpicture}
        \caption{Reachability}
    \end{figure}
\end{frame}
\note[itemize]{
    \item What is reachable from H if you follow the parent pointers?
}

\begin{frame}
    \frametitle{Reachability and Garbage Collection}
    \begin{figure}
        \centering
        \begin{tikzpicture}
            \gitDAG[grow right sep=2em]{
                A -- {{B, C -- D} -- H, {[nodes=unreachable] E -- F -- G}} --
                {[nodes=unreachable] I}
            };
        \end{tikzpicture}
        \caption{Reachable from H}
    \end{figure}
\end{frame}
\note[itemize]{
    \item Reachable from H
}

\begin{frame}
    \frametitle{Reachability and Garbage Collection}
    \begin{figure}
        \centering
        \begin{tikzpicture}
            \gitDAG[grow right sep=2em]{
                A -- {{[nodes=unreachable] {B, C -- D} -- H}, E -- F -- G} --
                {[nodes=unreachable] I}
            };
        \end{tikzpicture}
        \caption{Reachable from G}
    \end{figure}
\end{frame}
\note[itemize]{
    \item Reachable from G
}

\begin{frame}
    \frametitle{Reachability and Garbage Collection}
    \begin{figure}
        \centering
        \begin{tikzpicture}
            \gitDAG[grow right sep=2em]{
                A -- {{B, C -- D} -- H, E -- F -- G} -- I
            };
        \end{tikzpicture}
        \caption{Reachable from I}
    \end{figure}
\end{frame}
\note[itemize]{
    \item Everything on this graph is reachable from I
}

\begin{frame}[fragile]
    \frametitle{Reachability and Garbage Collection}
    \begin{figure}
        \centering
        \begin{tikzpicture}
            \gitDAG[grow right sep=2em]{
                A -- {B -- C' -- D', {[nodes=unreachable] C -- D}}
            };
            \gitbranch{master}{above=of B}{B}
            \oldgitbranch{foo}{right=of D}{D}
            \oldgitHEAD{right=of foo}{foo}
            \gitbranch{foo}{above=of D'}{D'}
            \gitHEAD{above=of foo}{foo}
        \end{tikzpicture}
        \caption{Rebase foo onto master}
    \end{figure}
    \begin{minted}[bgcolor=solarized-base2!50,frame=single,framesep=3pt]{console}
$ git rebase master
    \end{minted}
\end{frame}
\note[itemize]{
    \item Revisiting earlier rebase example
    \item Calling this rewriting history was a bit of a misnomer
    \item An alternate history C' and D' are created and \texttt{HEAD} and foo
          point to the alternate history
    \item The old history is still there. It is no longer referenced by foo
    \item \texttt{git log} doesn't show it
    \item \texttt{git log -all} doesn't show it
    \item But it's still in your database
    \item \texttt{git show <sha1sum>} of either C or D still works
    \item C and D won't be garbage collected for at least 30 days
    \item We'll come back to garbage collection later
}

\begin{frame}
    \frametitle{Recovering Your Work}
    Made a mistake?\\
    Need a change from several amends ago?\\
    Messed up a rebase?\\
    \begin{itemize}
        \item \texttt{git reflog}
        \item \texttt{git fsck}
        \item \texttt{git reset}
    \end{itemize}
\end{frame}
\note[itemize]{
    \item You can (almost) always recover your work
    \item If you've ever added a file, that snapshot is in your database
    \item We can recover it
}

\begin{frame}
    \frametitle{Lesson 8}
    \alert{Lesson 8}: Recovering Your Work
\end{frame}
\note[itemize]{
    \item \alert{Lesson 8}: Recovering Your Work
}
