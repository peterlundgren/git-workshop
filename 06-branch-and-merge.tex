\section[Section]{Branch and Merge}

\begin{frame}
    \frametitle{Branches}
    \centering
    \begin{figure}
        \begin{tikzpicture}
            \gitDAG[grow right sep=2em]{
                A
            };
            \gitbranch{master}{above=of A}{A}
        \end{tikzpicture}
        \caption{Branches are Pointers to Commits}
    \end{figure}
\end{frame}
\note[itemize]{
    \item By default, `git init` will create a master branch
    \item Most repositories have a master branch because most people are too
          lazy to change defaults
    \item Branches are pointers that point to commits
}

\begin{frame}
    \frametitle{\texttt{HEAD}}
    \centering
    \begin{figure}
        \begin{tikzpicture}
            \gitDAG[grow right sep=2em]{
                A
            };
            \gitbranch{master}{above=of A}{A}
            \gitHEAD{above=of master}{master}
        \end{tikzpicture}
        \caption{\texttt{HEAD} Points to Your Current Branch}
    \end{figure}
\end{frame}
\note[itemize]{
    \item \texttt{HEAD} points to current branch
    \item \texttt{HEAD} is what you have checked out on your filesytem
    \item \texttt{HEAD} is the parent of your next commit
}

\begin{frame}[fragile]
    \frametitle{\texttt{git branch}}
    \centering
    \begin{figure}
        \begin{tikzpicture}
            \gitDAG[grow right sep=2em]{
                A
            };
            \gitbranch{master}{above=of A}{A}
            \gitHEAD{above=of master}{master}
            \gitbranch{foo}{below=of A}{A}
        \end{tikzpicture}
        \caption{Creating a New Branch}
    \end{figure}
    \begin{minted}[bgcolor=solarized-base2!50,frame=single,framesep=0pt]{bash}
        $ git branch foo
    \end{minted}
\end{frame}
\note[itemize]{
    \item \texttt{HEAD} points to current branch
    \item \texttt{HEAD} is what you have checked out on your filesytem
    \item \texttt{HEAD} is the parent of your next commit
    \item Branches are cheap and fast. Writes 41 bytes to a file; that's it.
    \item \texttt{git branch foo} Creates a new branch called foo pointing to
          the same commit that HEAD is pointing to.
}

\begin{frame}[fragile]
    \frametitle{\texttt{git checkout}}
    \centering
    \begin{figure}
        \begin{tikzpicture}
            \gitDAG[grow right sep=2em]{
                A
            };
            \gitbranch{master}{above=of A}{A}
            \oldgitHEAD{above=of master}{master}
            \gitbranch{foo}{below=of A}{A}
            \gitHEAD{below=of foo}{foo}
        \end{tikzpicture}
        \caption{Switching Branches}
    \end{figure}
    \begin{minted}[bgcolor=solarized-base2!50,frame=single,framesep=0pt]{bash}
        $ git checkout foo
        $ git branch
        * foo
          master
    \end{minted}
\end{frame}
\note[itemize]{
    \item \texttt{git checkout} switches the current branch by changing what
          \texttt{HEAD} points to. If necessary, it will update your
          filesystem to match the commit pointed to by the branch.
    \item \texttt{git branch} will show you all of the local branches and put a
          star next to your current branch.
}

\begin{frame}[fragile]
    \frametitle{Make a Commit}
    \centering
    \begin{figure}
        \begin{tikzpicture}
            \gitDAG[grow right sep=2em]{
                A -- B
            };
            \gitbranch{master}{above=of A}{A}
            \oldgitbranch{foo}{below=of A}{A}
            \oldgitHEAD{below=of foo}{foo}
            \gitbranch{foo}{below=of B}{B}
            \gitHEAD{below=of foo}{foo}
        \end{tikzpicture}
        \caption{Make a Commit}
    \end{figure}
    \begin{minted}[bgcolor=solarized-base2!50,frame=single,framesep=0pt]{bash}
        $ git commit
    \end{minted}
\end{frame}
\note[itemize]{
    \item \texttt{git commit} Creates a new commit who's parent is whatever
          commit HEAD is pointing at. Then, it moves the branch \texttt{HEAD}
          is pointing at to the new commit.
    \item The only branch that moves is what \texttt{HEAD} points at.
    \item If you're ever scared about doing something, drop a branch behind. As
          long as you don't have a branch checked out, it's impossible to lose
          where it was.
}

\begin{frame}[fragile]
    \frametitle{Make Another Commit}
    \centering
    \begin{figure}
        \begin{tikzpicture}
            \gitDAG[grow right sep=2em]{
                A -- B -- C
            };
            \gitbranch{master}{above=of A}{A}
            \oldgitbranch{foo}{below=of B}{B}
            \oldgitHEAD{below=of foo}{foo}
            \gitbranch{foo}{below=of C}{C}
            \gitHEAD{below=of foo}{foo}
        \end{tikzpicture}
        \caption{Make Another Commit}
    \end{figure}
    \begin{minted}[bgcolor=solarized-base2!50,frame=single,framesep=0pt]{bash}
        $ git commit
    \end{minted}
\end{frame}

\begin{frame}[fragile]
    \frametitle{Checkout a New Branch}
    \centering
    \begin{figure}
        \begin{tikzpicture}
            \gitDAG[grow right sep=2em]{
                A -- B -- C
            };
            \gitbranch{master}{above=of A}{A}
            \gitbranch{bar}{below=of A}{A}
            \gitHEAD{below=of bar}{bar}
            \gitbranch{foo}{below=of C}{C}
            \oldgitHEAD{below=of foo}{foo}
        \end{tikzpicture}
        \caption{Checkout a New Branch}
    \end{figure}
    \begin{minted}[bgcolor=solarized-base2!50,frame=single,framesep=0pt]{bash}
        $ git checkout -b bar master
    \end{minted}
\end{frame}
\note[itemize]{
    \item \texttt{git checkout -b} is a shortcut for creating a branch and
          immediately checking it out.
}

\begin{frame}[fragile]
    \frametitle{Work on a New Branch}
    \centering
    \begin{figure}
        \begin{tikzpicture}
            \gitDAG[grow right sep=2em]{
                A -- {D -- E, B -- C}
            };
            \gitbranch{master}{above=of A}{A}
            \gitbranch{foo}{below=of C}{C}
            \gitbranch{bar}{above=of E}{E}
            \gitHEAD{above=of bar}{bar}
            \oldgitbranch{bar}{below=of A}{A}
            \oldgitHEAD{below=of bar}{bar}
        \end{tikzpicture}
        \caption{Work on a New Branch}
    \end{figure}
    \begin{minted}[bgcolor=solarized-base2!50,frame=single,framesep=0pt]{bash}
        $ git commit
        $ git commit
    \end{minted}
\end{frame}
\note[itemize]{
    \item Make two commits on branch bar
}

\begin{frame}[fragile]
    \frametitle{Merging}
    \centering
    \begin{figure}
        \begin{tikzpicture}
            \gitDAG[grow right sep=2em]{
                A -- {D -- E, B -- C}
            };
            \gitbranch{master}{above=of A}{A}
            \gitbranch{foo}{below=of C}{C}
            \gitbranch{bar}{above=of E}{E}
            \gitHEAD{above=of master}{master}
            \oldgitHEAD{above=of bar}{bar}
        \end{tikzpicture}
        \caption{Checkout master}
    \end{figure}
    \begin{minted}[bgcolor=solarized-base2!50,frame=single,framesep=0pt]{bash}
        $ git checkout master
    \end{minted}
\end{frame}
\note[itemize]{
    \item Checkout the branch you want to modify / merge into
    \item Switch back to the master branch
    \item Next, I want the master have the changes on the bar branch
    \item What should happen?
    \item Git checks to see if master is reachable from bar. If it is, it does
          the easiest possible thing.
}

\begin{frame}[fragile]
    \frametitle{Fast-Forward Merge}
    \centering
    \begin{figure}
        \begin{tikzpicture}
            \gitDAG[grow right sep=2em]{
                A -- {D -- E, B -- C}
            };
            \gitbranch{master}{above left=of E}{E}
            \gitbranch{foo}{below=of C}{C}
            \gitbranch{bar}{above=of E}{E}
            \gitHEAD{above=of master}{master}
            \oldgitbranch{master}{above=of A}{A}
            \oldgitHEAD{above=of master}{master}
        \end{tikzpicture}
        \caption{Fast-Forward Merge}
    \end{figure}
    \begin{minted}[bgcolor=solarized-base2!50,frame=single,framesep=0pt]{bash}
        $ git merge bar
    \end{minted}
\end{frame}
\note[itemize]{
    \item Want master took look like bar.
    \item It moves master up to the same commit that bar is at.
    \item Next, merge foo.
    \item What should happen?
    \item It's going to do a non-fast-forward merge. It has to create a new
          tree. The snapshot with both master and foo doesn't exist yet.
}

\begin{frame}[fragile]
    \frametitle{Non-Fast-Forward Merge}
    \centering
    \begin{figure}
        \begin{tikzpicture}
            \gitDAG[grow right sep=2em]{
                A -- {D -- E, B -- C} -- F
            };
            \gitbranch{master}{above=of F}{F}
            \gitbranch{foo}{below=of C}{C}
            \gitbranch{bar}{above=of E}{E}
            \gitHEAD{above=of master}{master}
            \oldgitbranch{master}{above left=of E}{E}
            \oldgitHEAD{above=of master}{master}
        \end{tikzpicture}
        \caption{Non-Fast-Forward Merge}
    \end{figure}
    \begin{minted}[bgcolor=solarized-base2!50,frame=single,framesep=0pt]{bash}
        $ git merge foo
    \end{minted}
\end{frame}
\note[itemize]{
    \item Git created a new snapshot, F, and moved master to it.
    \item F now has both the changes on foo and bar.
    \item Git encodes this by having a commit with two parents.
    \item Neither foo nor bar moved. Branches that are not checked out will not
          move.
}

\begin{frame}
    \frametitle{Lesson 2}
    \alert{Lesson 2}: Branching and Merging
\end{frame}
\note[itemize]{
    \item Demo lesson afterwards.
    \item cat .git/HEAD
    \item cat .git/refs/heads/feature/change-name
    \item Branches are just 41 byte files.
}

\begin{frame}
    \frametitle{Tags}
    Branches that Don't Move
\end{frame}
\note[itemize]{
    \item If you understand branches, then tags are easy.
}

\begin{frame}
    \frametitle{Lesson 3}
    \alert{Lesson 3}: Tags
\end{frame}
\note[itemize]{
    \item Start discussion: Lightweight vs. annotated tags.
}

\begin{frame}
    \frametitle{12 Everyday Commands}
    \begin{multicols}{3}
        \begin{itemize}
            \setlength\itemsep{3em}
            \item \alert{add}
            \item \alert{branch}
            \item \alert{checkout}
            \item \alert{commit}
            \item \alert{diff}
            \item fetch
            \item \alert{help}
            \item \alert{log}
            \item \alert{merge}
            \item push
            \item rebase
            \item \alert{status}
        \end{itemize}
    \end{multicols}
\end{frame}
\note[itemize]{
    \item We've added branch, checkout, and merge
}
